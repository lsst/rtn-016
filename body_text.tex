
\section{Introduction}
\label{sec:org2df4cd2}
\section{Tools, reports and their audiences}
\label{sec:org1244727}
\subsection{Introduction}
\label{sec:org8f7c0e7}
\subsubsection{Reports distinguished by frequency and intended audience}
\label{sec:org4f2efbe}
\subsubsection{Actors}
\label{sec:org0ce905c}
\begin{itemize}
\item Funding agencies
\item Rubin Observatory management
\item Science collaboration scientists
\item Observing scientists
\item Scheduler developers
\item The Community
\end{itemize}
\subsection{Published upcomming schedule}
\label{sec:org233906e}
\subsubsection{\href{https://ls.st/lse-61}{LST-61}/DMS-REQ-0353 Publishing predicted visit schedule}
\label{sec:orgb5b64ac}
\begin{quote}
Specification: A service shall be provided to publish to the community the next visit location and the predicted visit schedule provided by the OCS. This service shall consist of both a web page for human inspection and a web API to allow automated tools to respond promptly.

Discussion: The next visit and advanced schedule do not need to be published using the same service or protocol.
\end{quote}
\subsubsection{\href{https://ls.st/lse-30}{LSE-30}/OSS-REQ-0378 Advanced Publishing of Scheduler Sequence}
\label{sec:orgdf79be4}
\begin{quote}
The scheduling of the observing sequence lasting at least \texttt{schedSeqDuration} shall be published in advance of each observing visit.
\end{quote}
\subsection{Night reports}
\label{sec:orgf08954a}
\subsubsection{\href{https://ls.st/lse-30}{LSE-30}/OSS-REQ-0131 Nightly Summary Products}
\label{sec:orgf5a7795}
\begin{quote}
The Level 1 Data Products shall include a variety of reports, generated every night, that summarize the scientific quality of the Level 1 data (SDQA metrics), and the associated Observatory performance and performance of the Data Management subsystem.
\end{quote}
\subsubsection{\href{https://ls.st/lse-30}{LSE-30}/OSS-REQ-0406 Subsystem Nightly Reporting}
\label{sec:org9f211ee}
\begin{quote}
The LSST principal subsystems shall produce a searchable -interactive nightly report(s), from information in the EFD, summarizing per subsystem performance and behavior over a user defined period of time (e.g. the previous 24 hours).
\end{quote}
\subsubsection{\href{https://ls.st/lse-61}{LSE-61}/DMS-REQ-0096 Generate Data Quality Report Within Specified Time}
\label{sec:org7252852}
\begin{quote}
The DMS shall generate a nightly Data Quality Report within time dqReportComplTime in both human-readable and machine-readable forms.
\end{quote}
\subsubsection{\href{https://ls.st/lse-61}{LSE-61}/DMS-REQ-0097 Level 1 Data Quality Report Definition}
\label{sec:orgf49cfc4}
\begin{quote}
The DMS shall produce a Level 1 Data Quality Report that contains indicators of data quality that result from running the DMS pipelines, including at least: Photometric zero point vs. time for each utilized filter; Sky brightness vs. time for each utilized filter; seeing vs. time for each utilized filter; PSF parameters vs. time for each utilized filter; detection efficiency for point sources vs. mag for each utilized filter.
\end{quote}
\subsubsection{\href{https://ls.st/lse-61}{LSE-61}/DMS-REQ-0099 Level 1 Performance Report Definition}
\label{sec:org7263405}
\begin{quote}
The DMS shall produce a Level 1 Performance Report that provides indicators of how the DMS has performed in processing the night's observations, including at least: number of observations successfully processed through each pipeline; number of observations for each pipeline that had recoverable failures (with a record of the failure type and recovery mechanism); number of observations for each pipeline that had unrecoverable failures; number of observations archived at each DMS Facility; number of observations satisfying the science criteria for each active science program.
\end{quote}
\subsection{Monthly progress reports}
\label{sec:org15e06d3}
\subsubsection{\href{https://ls.st/lpm-73}{LPM-73}: LSST Operations Plan}
\label{sec:org49f2727}
section 8.3.3
\begin{quote}
\begin{itemize}
\item Tracking survey progress relative to the science requirements.
\item Optimizing the scheduled observations.
\item Balancing the observing schedule between survey operations, diagnostic activities, and calibration.
\end{itemize}
\end{quote}

\begin{quote}
Use the Operations Simulator toolbox to measure progress against the
survey plan. A written monthly progress report will be provided to
Headquarters, and weekly updates will be tracked in Chile.
Adjustments to the short term observing plan (choice of filters,
relative priority of science programs, etc.)  will be made in Chile.
If major changes to observing strategy appear to be required,
Headquarters will ask the PST for recommendations.
\end{quote}
\subsection{Periodic performance reviews}
\label{sec:org640357c}
\subsubsection{\href{https://ls.st/lse-29}{LSE-29}/LSR-REQ-0065 Survey performance reviews}
\label{sec:orga14519b}
\begin{quote}
The Observatory shall have the ability to provide periodic status
reports on the progress of the survey to allow both operations staff
and the community to assess the survey progress.
\end{quote}
\subsection{Common tools for collecting performance analysis}
\label{sec:orgeb21da6}
\subsubsection{\href{https://ls.st/lse-30}{LSE-30}/OSS-REQ-0067 Performance \& Trend Analysis Toolkit}
\label{sec:org7029949}
\begin{quote}
The LSST system shall provide a common tool kit for conducting performance analysis, including trending, on the telemetry captured in the Engineering \& Facility Database.
\end{quote}
\subsubsection{\href{https://ls.st/lse-29}{LSE-29}/LSR-REQ-0071 Scientific Oversight During Data Collection}
\label{sec:org82f28ad}
\begin{quote}
Requirement: The LSST Observatory shall be developed to allow an
observing scientist to have oversight of the Data Collection
process. This interaction shall be enabled either locally on the
summit or at remote locations. The data provided shall include all
observing condition data, telemetry data to assess telescope
conditions, and science data quality metrics for evaluation of the
data collection process.

Discussion: The objective of this requirement is to enable the
observing scientist to be involved directly in the observing
process. Under normal circumstances the observing scientist will not
intervene in the autonomous operations (LSR-REQ-0072), but should be
allowed to override if anomalous behavior occurs.
\end{quote}
\subsection{Tools for survey performance evaluation}
\label{sec:org53fc3e6}
\subsubsection{\href{https://ls.st/lse-29}{LSE-29}/LSR-REQ-0066 Survey performance evaluation}
\label{sec:org7a744ac}
\begin{quote}
The Project shall create the necessary survey performance evaluation
tools to predict the final results of the ten year survey based on the
actual survey completed to date, assess the impacts of survey strategy
changes resulting from changes in scientific priorities, and support
the planning of the survey on a variety of time scales, from nightly
through the entire 10 year duration.
\end{quote}
\subsubsection{\href{https://ls.st/lse-29}{LSE-29}/LSR-REQ-0070 Science Priorities and Survey Monitoring}
\label{sec:orge9968ea}
\begin{quote}
The LSST project shall monitor the scientific and technical progress
of the survey, communicate with the scientific user community and
establish survey priorities, and adjust the survey design as needed to
accomplish its goals given these priorities and achieved performance.
\end{quote}
\subsubsection{\href{https://ls.st/lse-30}{LSE-30}/OSS-REQ-0067 Performance \& Trend Analysis Toolkit}
\label{sec:orga7276ec}
\begin{quote}
The LSST system shall provide a common tool kit for conducting performance analysis, including trending, on the telemetry captured in the Engineering \& Facility Database.
\end{quote}
\subsection{Tools for interaction with the community}
\label{sec:org5d87304}
\subsubsection{\href{https://ls.st/lse-30}{LSE-30}/OSS-REQ-0033 Survey Planning and performance monitoring}
\label{sec:org14fb89e}
\begin{quote}
The LSST shall provide the tools and administrative processes
necessary to monitor the progress of the ongoing survey, provide
reports on the progress of the survey, respond to feedback from the
science community, and evaluate the impact of changing science
priorities over the 10 year survey lifetime.

Discussion:It is expected that the performance of this task will
require the use of detailed survey simulations in order to evaluate
scheduling alternatives and optimize the future performance of the
survey.
\end{quote}
\subsection{Interfaces for education and public outreach}
\label{sec:org20f84c4}
\subsubsection{\href{https://ls.st/lse-29}{LSE-29}/LSR-REQ-0113 EPO Products, Tools, and Interfaces}
\label{sec:orgddcc99f}
\begin{quote}
LSST EPO shall provide access to LSST data through tools, interfaces,
and learning experiences that are designedto engage communities with
different levels of knowledge, experience and skills.
\end{quote}
\subsubsection{\href{https://ls.st/lse-29}{LSE-29}/LSR-REQ-0116 EPO Fully Integrated}
\label{sec:orgf4096dd}
\subsection{Unspecified reports}
\label{sec:orgc808489}
\subsubsection{\href{https://ls.st/lse-30}{LSE-30}/OSS-REQ-0056 System Monitoring \& Diagnostics}
\label{sec:org4c7da18}
\subsubsection{\href{https://ls.st/lse-30}{LSE-30}/OSS-REQ-0063 System Monitoring \& Diagnostics Subsystem Metadata for Science Analysis}
\label{sec:org388f81a}
\subsubsection{\href{https://ls.st/lse-30}{LSE-30}/OSS-REQ-0068 Summit Environment Monitoring}
\label{sec:org2cfd71f}
\subsubsection{\href{https://ls.st/lse-30}{LSE-30}/OSS-REQ-0072 Weather and Meteorological Monitoring}
\label{sec:orgdab84d4}
\subsubsection{\href{https://ls.st/lse-30}{LSE-30}/OSS-REQ-0078 Maintenance Reporting}
\label{sec:orgfd202b9}
\subsubsection{\href{https://ls.st/lse-30}{LSE-30}/OSS-REQ-0079 Maintenance Tracking and Analysis}
\label{sec:org9f2e42c}
\subsubsection{\href{https://ls.st/lse-30}{LSE-30}/OSS-REQ-0314 Subsystem Performance Reporting}
\label{sec:org40711d6}
\begin{quote}
The LSST Observatory over the course of the 10-year survey shall monitor its performance with respect to its established baseline and report variances exceeding established thresholds.
\end{quote}

\section{Reports elements}
\label{sec:org2064e22}
\subsection{Numbers of exposures vs. time}
\label{sec:orgaf87b39}
\subsection{teff vs time}
\label{sec:orgf7328cc}
\subsection{Area at depths vs. time}
\label{sec:org86be2ae}
\subsection{Area times time at cadence}
\label{sec:orge1d9a9b}
\subsection{Total cadence maps}
\label{sec:orgc4a2248}
Maps of area on which we have achieved a given cadence for a given time.
\subsection{Active cadence maps}
\label{sec:orgdacea43}
\subsection{Current values of science metrics vs. time}
\label{sec:org42b9162}
\subsection{Extrapolated values of science metrics vs. time}
\label{sec:orgb8b6c07}
\subsection{Survey movies}
\label{sec:orga0225c4}
\subsubsection{Nightly}
\label{sec:org429ad03}
\subsubsection{Long-term}
\label{sec:org54cbe3a}
\subsection{Current area meeting cadence criteria}
\label{sec:orgb46ba0f}
\subsection{Numbers of objects detected by time, location, magnitude}
\label{sec:org68a9983}
\subsection{Scheduler feature maps}
\label{sec:org6cf2f01}
\subsection{Downtime (by whether expected and by cause)}
\label{sec:org8fba262}
\subsubsection{weather}
\label{sec:orgcc1d347}
\subsubsection{technical problems}
\label{sec:org84910c5}
\subsection{Time spent on different programs (FWD, DDF, ToO, etc.)}
\label{sec:org1a662a9}
\subsection{Overhead from instrument performance}
\label{sec:org9dcc1b2}
\subsubsection{slew time}
\label{sec:org610f156}
\subsubsection{filter change time}
\label{sec:org3f15a04}
\subsubsection{readout time}
\label{sec:orgc8f69fd}
\subsubsection{total efficiency (exposure time/night time)}
\label{sec:org489b0e0}
\subsection{Slew angle distributions}
\label{sec:org750750b}
\subsection{Airmass distribution vs. time}
\label{sec:orgf5364a4}
\subsection{Airmass distribution vs. location on sky}
\label{sec:org61828c5}
\subsection{Delivered seeing distribution}
\label{sec:orgd969073}
\subsection{Seeing achieved vs expected}
\label{sec:org8c30700}
\subsection{Sky brightness achieved vs expected}
\label{sec:org1bc45fb}
\subsection{Weather (clouds) achieved vs expected}
\label{sec:orgf2f2124}
\subsection{DDF cadence evaluation plots}
\label{sec:org53f0866}
\subsection{WFD cadence evaluation maps}
\label{sec:org2c5efc8}
\subsection{Accuracy of published next visit and advanced schedule information}
\label{sec:orga4e2b9f}
\subsection{Number of filter changes}
\label{sec:org4507f90}
\subsection{Fraction of time scheduled in blobs}
\label{sec:orgd0aa533}
\subsection{Current season length maps}
\label{sec:org6a5eb51}
\subsection{Maximum gap in season maps}
\label{sec:orgec320cb}
\subsection{Time since last observation maps}
\label{sec:org83f80bb}
\subsection{Retrospective simulation}
\label{sec:orgf491c31}
If we simulated the time we just did, do we get what we actually got?
\section{Infrastructure}
\label{sec:org6cbd899}
\subsection{Suppliers of required data}
\label{sec:orgda99e9c}
\subsubsection{Engineering and Facilities Database (EFD)}
\label{sec:orgbb37c6f}
\begin{itemize}
\item \href{https://ls.st/LTS-210}{LTS-210: Engineering and Facility Database Design Document}
\end{itemize}
\subsubsection{DM Butler}
\label{sec:orgbe7ee1c}
\subsection{Reporting infrastructure}
\label{sec:org0c0d061}
\subsubsection{SQuaRE}
\label{sec:org6c937dc}
\subsubsection{flux}
\label{sec:orgaa20903}
\subsection{Specified reports and monitoring tools}
\label{sec:org5de654b}
