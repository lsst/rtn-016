
\subsection{Science project planning (static)}
\label{sec:org3342502}
\begin{description}
\item[{Purpose}] To determine when (or if) survey co-add data will be useful for a science project requiring a certain area at a certain depth.
\item[{Context}] A small group of astronomers comes up with a brilliant idea for a project, and conclude that they need a survey that covers at least 10000 square degrees and has \(5-\sigma\) limiting magnitudes of at least 26.0 in r and 25.0 in z. They want to know when Rubin Observatory might have collected the necessary data to construct such co-adds.
\item[{Primary actor}] Astronomers planning a project.
\item[{Trigger}] A group of astronomers has an inspiration for a project.
\item[{Success conditions}] The collaboration of astronomers has a good understanding of when their project will become feasible, or to what degree.
\item[{Main success scenario}] The group might take the following steps:
\begin{enumerate}
\item The astronomers retrieve plots of \(t_{\mbox{\tiny eff}}\) vs. calendar date for the filters for relevance from a recent night report or quarterly report.
\item They note that in most years, the slope of accumulated \(t_{\mbox{\tiny eff}}\) followed the baseline, but there was one year during which progress was more shallow, after which it returned to the nominal slope.
\item They estimate the date on which the \(t_{\mbox{\tiny eff}}\) corresponds to the magnitudes of interest for each filter, extrapolating from the latest point by eye, using the typical slope. They also use the maximum and minimum slopes (visually obvious from inspection of the plot) to estimate optimistic and pessimistic dates on which the data might have been collected.
\end{enumerate}
\end{description}
\subsection{Deep Drilling Field scheduler evaluation}
\label{sec:orgd5c6a03}
\begin{description}
\item[{Purpose}] To evaluate the scheduler's performance in scheduling DDF fields.
\item[{Context}] The Deep Drilling Fields (DDFs) mini-survey consists of a set of additional visits at a handful of pointings. These additional visits will result in a denser cadence at the selected pointings, enabling the study of variable and transient objects over longer seasons and with a denser cadence. Furthermore, these additional visits will result in a much deeper co-add limiting magnitude for the limited area they cover. The success of the deep drilling mini-survey depends on the observing cadence, season length, and overall depths in these fields. Additionally, optimization of the DDF mini-survey depends on the same kinds of features all exposures do, including observation close to transit and when the sky brightness is as low as possible given restrictions imposed by the required cadence.
\item[{Primary actor}] The observatory support scientists and the scheduling team will both evaluate the scheduler's performance on DDF fields on a routine basis.
\item[{Trigger}] Evaluation of DDF scheduler performance may be triggered by any of several conditions, including:
\begin{itemize}
\item preparation for the pre-night review, during which the predicted DDF visits generated by the simulator require sanity checking.
\item preparation of night reports, so that the performance of the scheduler on DDF fields during the just completed night can be checked.
\item evaluation of scheduler changes, particularly proposed changes to DDF scheduler strategy.
\item preparation of survey progress reports. The DDF program is important for the success of LSST as a whole, and evaluation of progress observing DDFs should be included in the global progress reports.
\end{itemize}
\item[{Success conditions}] If successful, the actors in this use cases will be able to tell whether DDF fields were scheduled at appropriate times, including identifying nights when they should have been scheduled by were not, nights when they should not have been scheduled by were, and nights during which visits on DDF fields were scheduled at a different time than optimal.
\item[{Main success scenario}] The agent will need to produce and examine a collection of different diagnostics, including:
\begin{itemize}
\item a cadence plot, showing the nights on which each DDF field was observed, and the depths of the images obtained. Figure \ref{fig:org57753e9} shows a sample cadence plot.
\item a time use hourglass plot, showing the nights on which each DDF field was observed, and the time it was observed relative to transit. Figure \ref{fig:orgb90ffed} shows an example time use hourglass plot.
\item rule-based condition checks. Several sciences collaborations have requested rule-based DDF strategies (e.g. each DDF field should be observed on a cadence of X days and a season length of Y days, at a minimum). Reports should flag any nights on which the rules adopted by the project are violated.
\item metric based checks. When triggered by new strategies or creation of global reports, the values of science metrics on each DDF field should be reported. For evaluation of DDF scheduling for individual nights (including the pre-night review and night reports), science metrics may be evaluated on simulations with and without DDF visits, and the effects on final science metrics and fraction of time taken by DDFs evaluated.
\end{itemize}
\end{description}
\subsection{Other Mini-survey evaluation}
\label{sec:orgc05a452}
\begin{description}
\item[{Purpose}] To estimate the effectiveness of an ongoing target of opportunity program (or other mini-survey), and its impact on the progress made on WFD science goals and other mini-surveys.
\item[{Context}] Current plans dedicate a fraction of Rubin Observatory observing time to mini-surveys. Estimation of the impact of these surveys on the WFD and other programs will not always be straightforward: visits scheduled for them will often also contribute to WFD goals, so even though the time taken for them may not be fully optimal for WFD science, it is not fully lost to it either. Furthermore, the effectiveness of the mini-survey must itself be evaluated for how well it progressing towards its own scientific goals. If it is making little progress but having a significant impact on WFD progress, the mini-survey observing strategy may need to be modified, or the mini-survey discontinued.
\item[{Primary actor}] the survey cadence optimization committee (SCOC)
\item[{Trigger}] Regular meeting of the SCOC.
\item[{Success conditions}] The SCOC has the data necessary to make informed decisions on renewing mini-surveys.
\item[{Main success scenario}] \begin{enumerate}
\item The SCOC examines the most recent survey progress report and reports from the science collaborations on status of the science from the mini-survey.
\item By comparing WFD science metrics actually achieved to those that opsim simulations report would have happened had the mini-survey not taken place, the SCOC determines whether the mini-survey's impact on the FWD was well estimated, or if it was more or less disruptive than expected.
\item The SCOC recommends the continuance, alteration, or conclusion of the mini-survey. For example, it may apply additional conditions on a target of opportunity mini-survey if triggered visits were less useful for the FWD than had been expected. Alternately, if a mini-survey has been under-performing on its own scientific metrics, but alterations to survey strategy have had little or no impact on WFD science goals, the SCOC may recommend that the mini-survey be expanded.
\end{enumerate}
\end{description}
