\documentclass[DM,authoryear,toc]{lsstdoc}
% lsstdoc documentation: https://lsst-texmf.lsst.io/lsstdoc.html
\input{meta}

% Package imports go here.

% Local commands go here.

%If you want glossaries
%\input{aglossary.tex}
%\makeglossaries

\title{Background and concepts for monitoring survey progress and scheduler performance}

% Optional subtitle
% \setDocSubtitle{A subtitle}

\author{%
Eric Neilsen
}

\setDocRef{RTN-016}
\setDocUpstreamLocation{\url{https://github.com/lsst/rtn-016}}

\date{\vcsDate}

% Optional: name of the document's curator
% \setDocCurator{The Curator of this Document}

\setDocAbstract{%
Information useful for planning the development of software and processes required to report, monitor, and track observing progress and scheduler performance for the Legacy Survey of Space and Time (LSST) by the Rubin Observatory is collected.
A handful of example use-case descriptions provide illustrative examples of what kinds of report elements (plots and metrics) will be useful to whom, and why.
More extensive catalogues of stakeholders and candidate reports and report elements follow.
A description of existing and planned Rubin Observatory infrastructure outlines the operating environment and technical resources available for creating these reports.
}

% Change history defined here.
% Order: oldest first.
% Fields: VERSION, DATE, DESCRIPTION, OWNER NAME.
% See LPM-51 for version number policy.
\setDocChangeRecord{%
  \addtohist{1}{YYYY-MM-DD}{Unreleased.}{Eric Neilsen}
}


\begin{document}

% Create the title page.
\maketitle
% Frequently for a technote we do not want a title page  uncomment this to remove the title page and changelog.
% use \mkshorttitle to remove the extra pages

% ADD CONTENT HERE
% You can also use the \input command to include several content files.

\appendix
% Include all the relevant bib files.
% https://lsst-texmf.lsst.io/lsstdoc.html#bibliographies
\section{References} \label{sec:bib}
\renewcommand{\refname}{} % Suppress default Bibliography section
\bibliography{local,lsst,lsst-dm,refs_ads,refs,books}

% Make sure lsst-texmf/bin/generateAcronyms.py is in your path
\section{Acronyms} \label{sec:acronyms}
\addtocounter{table}{-1}
\begin{longtable}{p{0.145\textwidth}p{0.8\textwidth}}\hline
\textbf{Acronym} & \textbf{Description}  \\\hline

API & Application Programming Interface \\\hline
DDF & Deep Drilling Fields \\\hline
DES & Dark Energy Survey \\\hline
DIMM & Differential Image Motion Monitor \\\hline
DM & Data Management \\\hline
DMS & Data Management Subsystem \\\hline
DMS-REQ & Data Management System Requirements prefix \\\hline
DMTN & DM Technical Note \\\hline
EFD & Engineering and Facility Database \\\hline
EPO & Education and Public Outreach \\\hline
FoM & Figure of Merit \\\hline
IR & infrared \\\hline
LOVE & LSST Operations Visualization Environment \\\hline
LPM & LSST Project Management (Document Handle) \\\hline
LSE & LSST Systems Engineering (Document Handle) \\\hline
LSR & LSST System Requirements; LSE-29 \\\hline
LSST & Legacy Survey of Space and Time (formerly Large Synoptic Survey Telescope) \\\hline
LTS & LSST Telescope and Site  (Document Handle) \\\hline
OCS & Observatory Control System \\\hline
OSS & Observatory System Specifications; LSE-30 \\\hline
PSF & Point Spread Function \\\hline
PST & Project Science Team \\\hline
PSTN & Project Science Technical Note \\\hline
RSS & square root of the sum of the squares \\\hline
RTN & Rubin Technical Note \\\hline
SCOC & Survey Cadence Optimization Committee \\\hline
SDSS & Sloan Digital Sky Survey \\\hline
SEWG & Survey Evaluation Working Group \\\hline
SQR & SQuARE document handle \\\hline
SQuaRE & Science Quality and Reliability Engineering \\\hline
SQuaSH & Science Quality Analysis Harness \\\hline
WFD & Wide Fast Deep \\\hline
ZTF & Zwicky Transient Facility \\\hline
\end{longtable}

% If you want glossary uncomment below -- comment out the two lines above
%\printglossaries





\end{document}
